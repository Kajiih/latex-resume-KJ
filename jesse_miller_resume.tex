%-------------------------
% Resume in Latex
% Author : Jesse Miller
%------------------------

\documentclass[letterpaper,11pt]{article}
\usepackage[
    left=0.5in,
    right=0.5in,
    top=0.5in,
    bottom=0.5in,
    % headheight=15pt,
    % footskip=20pt
]{geometry}

\usepackage{latexsym}
% \usepackage[empty]{fullpage}
\usepackage{titlesec}
\usepackage{marvosym}
\usepackage[usenames,dvipsnames]{color}
\usepackage{verbatim}
\usepackage{enumitem}
\usepackage[hidelinks]{hyperref}
\usepackage{fancyhdr}
\usepackage[english]{babel}
\usepackage{tabularx}
\usepackage{hyphenat}
\usepackage{fontawesome}
\usepackage{enumitem}
\input{glyphtounicode}


%---------- FONT OPTIONS ----------
% sans-serif
% \usepackage[sfdefault]{FiraSans}
% \usepackage[sfdefault]{roboto}
% \usepackage[sfdefault]{noto-sans}
% \usepackage[default]{sourcesanspro}

% serif
% \usepackage{CormorantGaramond}
% \usepackage{charter}


\pagestyle{fancy}
\fancyhf{} % clear all header and footer fields
\fancyfoot{}
\renewcommand{\headrulewidth}{0pt}
\renewcommand{\footrulewidth}{0pt}
% Set the footskip to a reasonable value
\setlength{\footskip}{25pt}

% Adjust margins
% \addtolength{\oddsidemargin}{-0.5in}
% \addtolength{\evensidemargin}{-0.5in}
% \addtolength{\textwidth}{1in}
% \addtolength{\topmargin}{-.5in}
% \addtolength{\textheight}{1.0in}

\urlstyle{same}

\raggedbottom
\raggedright
\setlength{\tabcolsep}{0in}

% Sections formatting
\titleformat{\section}{
  \vspace{-10pt}\scshape\raggedright\large
}{}{0em}{}[\color{black}\titlerule \vspace{-5pt}]

% Ensure that generate pdf is machine readable/ATS parsable
\pdfgentounicode=1

% \linespread{1.0}

\setlist[itemize,1]{label=$\vcenter{\hbox{\tiny$\bullet$}}$} 
\setlist[itemize,2]{label=$\vcenter{\hbox{\tiny$\bullet$}}$}
\setlist[itemize,3]{label=$\vcenter{\hbox{\tiny$\circ$}}$}

%-------------------------
% Custom commands

\newcommand{\resumeItem}[1]{
  \item\small{
    {#1}\vspace{-4pt}
  }
}

% company name -- location
% job title -- date
\newcommand{\resumeSubheading}[4]{
  \vspace{-2pt}\item
    \begin{tabular*}{1\textwidth}[t]{l@{\extracolsep{\fill}}r}
      \textbf{#1} & #2 \\
      \textit{\textbf{\small#3}} & \textit{\small #4} \\
    \end{tabular*}\vspace{-7pt}
}

% job title -- date
\newcommand{\resumeSubSubheading}[2]{
    \vspace{-2pt}\item
    \begin{tabular*}{1.0\textwidth}{l@{\extracolsep{\fill}}r}
      \textit{\textbf{\small#1}} & \textit{\small #2} \\
    \end{tabular*}\vspace{-7pt}
}


\newcommand{\resumeEducationHeading}[6]{
  \vspace{-2pt}\item
    \begin{tabular*}{1.0\textwidth}[t]{l@{\extracolsep{\fill}}r}
      \textbf{#1} & #2 \\
      \textit{\small#3} & \textit{\small #4} \\
      \textit{\small#5} & \textit{\small #6} \\
    \end{tabular*}\vspace{-5pt}
}


\newcommand{\resumeProjectHeading}[2]{
    \vspace{-2pt}\item
    \begin{tabular*}{1.0\textwidth}{l@{\extracolsep{\fill}}r}
      \small#1 & #2 \\
    \end{tabular*}\vspace{-7pt}
}


\newcommand{\resumeOrganizationHeading}[4]{
  \vspace{-2pt}\item
    \begin{tabular*}{1.0\textwidth}[t]{l@{\extracolsep{\fill}}r}
      \textbf{#1} & \textit{\small #2} \\
      \textit{\small#3}
    \end{tabular*}\vspace{-7pt}
}

\newcommand{\resumeSubItem}[1]{\resumeItem{#1}\vspace{-4pt}}

\newcommand{\resumeSubHeadingListStart}{\begin{itemize}[leftmargin=0.0in, label={}]}
\newcommand{\resumeSubHeadingListEnd}{\end{itemize}}
\newcommand{\resumeItemListStart}{\begin{itemize}[leftmargin=15pt]}
\newcommand{\resumeItemListEnd}{\end{itemize}\vspace{-5pt}}

\newcommand{\bref}[2]{\href{#1}{\color{blue}{#2}}}

%-------------------------------------------
%%%%%%  RESUME STARTS HERE  %%%%%%%%%%%%%%%%%%%%%%%%%%%%


\begin{document}

%---------- HEADING ----------
\def\spaceAfterLogo{0.5pt}
\begin{center}
  \textbf{\Huge Jesse Miller} \\ \vspace{3pt}
  \small
  \faMobile \hspace{\spaceAfterLogo} \href{tel:8454281532}{(845) 428-1532}
  $|$
  \faEnvelope \hspace{\spaceAfterLogo} \href{mailto:jam643@cornell.edu}{jam643@cornell.edu}
  $|$
  \faLinkedinSquare \hspace{\spaceAfterLogo} \href{https://www.linkedin.com/in/jam643}{linkedin.com/in/jam643}
  $|$
  \faGithub \hspace{\spaceAfterLogo} \href{https://github.com/jam643}{github.com/jam643}
  % $|$
  % \faGlobe \hspace{\spaceAfterLogo} \href{https://jam643.github.io}{Portfolio}
  $|$
  \faMapMarker \hspace{\spaceAfterLogo} \href{https://maps.app.goo.gl/hpB8QGkNWpEHrJGD9}{Cambdrige, MA}
\end{center}


%----------- EXPERIENCE -----------

\section{Experience}
\resumeSubHeadingListStart

\resumeSubheading
{Motional}{Boston, MA}
{Principal Robotics Control Engineer}{May 2024 \textbf{--} Present}
\resumeItemListStart
\resumeItem{Designed, trained, and deployed a \textbf{PPO-based RL Planner} using PyTorch which outperformed the company’s imitation learning based ML planner in comfort and safety in most scenarios in sim and on-vehicle.}
\resumeItem{\textbf{Led a cross-org Reflexive Planner project}, a low latency task for collision prevention critical to achieving the Driverless milestone. Coded features that \textbf{reduced false-positive rate by 400\%} and designed regression tests and metrics for performance monitoring. Presented results directly to \textbf{Motional's CTO}.}
\resumeItem{Awarded Motional's monthly \textbf{"Mobilizer Award"} and consistently recieved "exceeds expectation" for every performance review.}
\resumeItemListEnd

\resumeSubSubheading
{Senior Robotics Control Engineer and Team Lead}{August 2020 \textbf{--} May 2024}
\resumeItemListStart
\resumeItem{Code-owner and developer for a \textbf{nonlinear trajectory optimization} formulated with FORCES Pro in Python and deployed in C++.}
\resumeItemListStart
\resumeItem{Collaborated to implement and \textbf{patent a novel Contingency MPC} approach for handling multimodal predictions in trajectory optimizer.}
\resumeItem{Invented a novel algorithm for generating assertive lane change trajectories across planning control stack; deployed on AV.}
\resumeItem{Tuned controller performance on-vehicle, contributing to investor analysts’ feedback that ride was \textbf{“smoother than Waymo’s”}.}
\resumeItemListEnd
\resumeItem{Led Path Tracker Perfomance team (6 ppl) creating Python and SQL metric frameworks, automating release tests, and improving LQR tracking performance. Reduced rate of high cross-track error from \textbf{25 to 1 violation per 1K miles}.}
\resumeItem{Collaborated with team to formulate and implement a \textbf{convex trajectory optimizer} using OSQP, reducing latency by 50\%.}
\resumeItem{Developed MySQL metric dashboards for visualization, triage, and drilldown of controls related errors; monitored weekly.}
\resumeItem{\textbf{Led Trajectory Formulation Team} for a quarter; defined team’s north star strategy, conducted one-on-ones, and oversaw successful demo deployment.}
\resumeItem{Developed and patented an \textbf{ML based vehicle dynamics model} for simulation (LSTM in PyTorch) and deployed it in C++, reducing avg displacement error by 40\% compared to physics based model.}
\resumeItem{Primary code owner and developer for repo containing Python analysis scripts used for AV log visualization and debug, regularly used by 30+ people.}
\resumeItemListEnd

\resumeSubSubheading
{Robotics Control Engineer III}{October 2018 \textbf{--} August 2020}
\resumeItemListStart
\resumeItem{Owner of a dynamic bicycle model based \textbf{path tracker module}, refactoring and integrating code during company merger.}
\resumeItem{Designed and coded ‘Mock Planner’ \& ‘Fault Injection’ tools in C++ and Python used to stress test controller \& for vehicle operator training.}
\resumeItem{Curated a set of ride comfort metrics (e.g. motion sickness likelihood), implementing extractors to monitor comfort trends.}
\resumeItem{Mentored intern who explored trajectory optimizer solver modifications for improved latency and performance.}
\resumeItemListEnd

\resumeSubSubheading
{Autonomous Vehicle Engineer}{August 2017 \textbf{--} October 2018}
\resumeItemListStart
\resumeItem{Worked on vehicle integration and sensor calibration, outfitting the Renault Zoe EVs with nuTonomy’s AV compute stack.}
\resumeItem{Streamlined AV closed course testing, curating a weekly KPI report on AV performance shared with C-Suite.}
\resumeItemListEnd

\resumeSubheading
{Sandia National Labs}{Albuquerque, NM}
{R\&D Electromechanical Engineer and Technical Lead}{August 2016 \textbf{--} August 2017}
\resumeItemListStart
\resumeItem{Tuned a PID controller for a DC motor system using system identification and control design methods in MATLAB/Simulink.}
\resumeItem{Developed a tester using fiber optics, PXI, LabView, 3D printing, and control design to quantify a mechanism’s performance.}
\resumeItem{Designed tests and analyzed results to characterize the performance of various mechanisms relating to nuclear weapon safety.}
\resumeItem{Modeled 3D multibody rigid body dynamic mechanism using Siemens LMS software and performed tests to verify the model.}
\resumeItemListEnd

\resumeSubheading
{Autonomous Sailboat Project}{Ithaca, NY}
{Master’s Thesis, Dynamics Subteam Lead}{January 2015 \textbf{--} May 2016}
\resumeItemListStart
\resumeItem{Created \bref{https://github.com/jam643/SailboatSim3D/tree/master}{a simulation framework} to model the 3D dynamics of a sailboat in MATLAB to determine the feasibility of autonomously sailing around the world.}
\resumeItem{Wrote \bref{https://bpb-us-w2.wpmucdn.com/sites.coecis.cornell.edu/dist/5/91/files/2017/06/Semester-Report-Spring-2016-Jesse-Miller-1e485cb.pdf}{a thesis paper} describing a novel ‘rudder connected to sail’ design to improve the directional stability of a robotic sailboat.}
\resumeItem{Presented findings at \bref{https://robotics.cornell.edu/2019/08/14/a-directionally-self-stable-robotic-sail-boat-concept-and-simulations/}{a robotics seminar} with advisor \bref{http://ruina.tam.cornell.edu/}{Andy Ruina} at Cornell University and received \bref{https://www.mae.cornell.edu/news/sibley-school-mechanical-aerospace-engineering-student-awards-0}{the Kelly Prize} for aerospace related research.}
\resumeItem{Provided recommended robotic sailboat design to \bref{https://cusail.com/}{CUSail project team} at Cornell which they built and tested on Cayuga Lake, verifying the simulation results.}
\resumeItemListEnd

\resumeSubheading
{Engineering Dynamics Class (MAE 2030)}{Ithaca, NY}
{Graduate Teaching Specialist}{January 2016 \textbf{--} May 2016}
\resumeItemListStart
\resumeItem{Taught two recitation sections with about 15 students in each, held regular office hours, and graded exams.}
\resumeItem{Volunteered to fill in for the Professor, teaching a one-and-a-half-hour lecture to about 150 students.}
\resumeItem{Received \textbf{4.88/5 overall rating} on my TA evaluation based on 16 student responses and was nominated to be a TA trainer.}
\resumeItemListEnd

\resumeSubheading
{Amazon Robotics}{North Reading, MA}
{Data Analyst and Performance Intern}{May 2015 \textbf{--} August 2015}
\resumeItemListStart
\resumeItem{Created a random forest machine learning predictor using Python and MySQL to predict the time it takes to pick an item off a shelf.}
\resumeItem{Designed a Linux service using Java and XML, allowing the predictor to be easily used.}
\resumeItem{Implemented the predictor service with a performance website, improving metrics used to assess employee performance.}
\resumeItemListEnd

\resumeSubheading
{Innovative Scientific Solutions Inc.}{Dayton, OH}
{Engineering Co-op Student}{January 2014 \textbf{--} August 2014}
\resumeItemListStart
\resumeItem{\textbf{Invented} a unique idea to listen to the sounds produced by handwriting and understand what was written.}
\resumeItem{Developed and \bref{https://github.com/jam643/WriteHear/tree/master}{coded a MATLAB program} to transcribe handwriting by only using a microphone.}
\resumeItem{\bref{https://github.com/jam643/WriteHear/blob/master/WriteHear.pdf}{Presented my invention} in a \textbf{U.S. Air Force wide competition} against 37 other inventions.}
\resumeItem{\textbf{Won} the competition and received \$15,000 in research funding and a patent for the invention.}
\resumeItem{Designed in NX Unigraphics and performed an experiment to test the effects of unsteady flow through an intercooler.}
\resumeItem{Published and presented \bref{https://www.researchgate.net/publication/290192705_Effect_of_Unsteady_Flow_on_Intercooler_Performance}{a conference paper} based on these findings in the SAE 2014 Aerospace Conference.}
\resumeItemListEnd

\resumeSubheading
{Autonomous Systems Laboratory}{Ithaca, NY}
{Mechanical Engineering Researcher}{August 2014 \textbf{--} December 2014}
\resumeItemListStart
\resumeItem{Developed and wrote Python handlers to control a robotic ball, Sphero, with a path planning program, LTLMoP.}
\resumeItem{Successfully demoed the code, having Sphero autonomously traverse a map and react to its environment.}
\resumeItemListEnd

\resumeSubHeadingListEnd


%----------- EDUCATION -----------
\section{Education}
\resumeSubHeadingListStart

\resumeSubheading
{Cornell University, College of Engineering $|$ \small{Ithaca, NY}}{\textit{August 2015 \textbf{--} May 2016}}
{Master of Engineering Robotics, Controls, and Dynamics}{\textbf{GPA: 4.17/4.0}}
\resumeItemListStart
\resumeItem{\bref{https://www.mae.cornell.edu/news/sibley-school-mechanical-aerospace-engineering-student-awards-0}{Outstanding Achievement Award}: Given to top two MechE Masters students with the \textbf{highest academic standing}.}
\resumeItem{\bref{https://www.mae.cornell.edu/news/sibley-school-mechanical-aerospace-engineering-student-awards-0}{Kelly Prize}: Excellence in aerospace engineering, recieved for research on a novel robotic sailboat design.}
\resumeItem{\textbf{Relevant Courses:} Autonomous Mobile Robots, Robot Motion, Feedback Control, Multivar Control Theory, Advanced Dynamics.}
\resumeItemListEnd

\resumeSubheading
{Cornell University, College of Engineering $|$ \small{Ithaca, NY}}{\textit{August 2011 \textbf{--} May 2015}}
{Bachelor of Science in Mechanical Engineering}{\textbf{Major GPA 4.24/4.0 $|$ Cumulative GPA: 4.15/4.0}}
\resumeItemListStart
\resumeItem{\bref{https://www.mae.cornell.edu/news/sibley-school-mechanical-aerospace-engineering-student-awards-1}{Sibley Prize}: Awarded to the two mechanical engineering graduating seniors with the \textbf{highest overall GPA}.}
\resumeItem{\bref{https://www.mae.cornell.edu/news/sibley-school-mechanical-aerospace-engineering-student-awards-1}{Frank O. Ellenwood Prize}: Awarded to the M.E. students with the \textbf{highest GPA} in heat and power courses.}
\resumeItemListEnd

% \resumeSubheading
% {Tri-Valley High School $|$ \small{Grahamsville, NY}}{\textit{September 2007 \textbf{--} May 2011}}
% {Valedictorian}{}

\resumeSubHeadingListEnd


%----------- SELECT PATENTS -----------

\section{Select Patents}
\begin{itemize}[leftmargin=15pt]
  \tiny\resumeItem{\textbf{Autonomous Driving Mode Engagement}, US patent US20230159063A1, March 25, 2023. I developed a method for comfortable and safe transition from manual to autonomous vehicle control.}
  \tiny\resumeItem{\textbf{Control Architectures for Autonomous Vehicles}, US patent US11794775B2, October 24, 2023. I collaborated to develop and implement novel trajectory optimization based control methods.}
  \tiny\resumeItem{\textbf{ML Based Vehicle Dynamic Simulator Model} (patent pending), Docket No. I2023099, May 24, 2023. I helped design and deploy an LSTM model trained on vehicle dynamics.}
  \tiny\resumeItem{\textbf{Shortened Contingency Horizon MPC With a Terminal Car-Follow Constraint} (patent pending), Docket No. I2023017, January 27, 2023. I designed and implemented a sparse contingency trajectory optimization formulation that improved solve-time while maintaining recursive safety guarantees.}
  \tiny\resumeItem{\textbf{Autonomous Vehicle with Contingency Consideration in Trajectory Realization} (patent pending), Application No US18164652, February 6, 2023. I co-designed and implemented a joint trajectory optimization formulation for safely and comfortably handling multimodal predictions.}
\end{itemize}\vspace{-5pt}

%----------- SKILLS -----------

\section{Skills \& Interests}
\resumeSubHeadingListStart
\small{\item{

              \textbf{Languages:}{ Python, C++, MATLAB, SQL, LabVIEW, C} \\ \vspace{3pt}

              \textbf{Technologies:}{ PyTorch, FORCES Pro, Stable-Baselines3, OSQP, Ubuntu/Linux, Git, CMake, Bazel, \LaTeX, Jupyter notebook, Simulink, OpenCV, Raspberry Pi, Arduino, SolidWorks, Blender} \\ \vspace{3pt}

              \textbf{Continued Education:}{ \bref{https://github.com/jam643/CS285DeepRL/tree/master}{Deep Reinforcement Learning CS285}, \bref{https://github.com/jam643/UnderactuatedRobotics-6p832-JesseM}{Underactuated Robotics MIT 6.832}, Udemy Design Patterns in C++} \\ \vspace{3pt}

              \textbf{Interests:}{ Acoustic guitar, ping pong, badminton, tennis} \\ \vspace{3pt}

        }}
\resumeSubHeadingListEnd

%----------- PROJECTS -----------

\section{Projects}
\resumeSubHeadingListStart

\resumeProjectHeading
{\textbf{Gamified Vehicle Path Tracking} $|$ \emph{\bref{https://github.com/jam643/TheTrolleyProblemGame}{GitHub \faGithub}}}{}
\resumeItemListStart
\resumeItem{I implemented a variety of tunable path tracking algorithms (e.g. LQR, Stanley, Pure Pursuit) in Python, wrapped in a gamified dynamic vehicle simulator. The user draws a b-spline path guiding the vehicle through a series of obstacles. The user can interactively tune the controllers and view Jupyter notebooks with derivations of the algorithms.}
\resumeItemListEnd

% \resumeProjectHeading
% {\textbf{Autonomous Sailboat 3D Dynamics Simulator} $|$ \emph{\bref{https://github.com/jam643/SailboatSim3D}{GitHub \faGithub}}}{}
% \resumeItemListStart
% \resumeItem{A 3D dynamics simulation of a robotic sailboat in MATLAB I developed as part of my Master's thesis at Cornell. The simulator was used to verify the passive stability properties of a novel robotic sailboat design as well as optimize the design for the boat that was later built by \bref{https://cusail.com/}{a project team, CUSail}, at Cornell.}
% \resumeItemListEnd

% \resumeProjectHeading
% {\textbf{WriteHear} $|$ \emph{\bref{https://github.com/jam643/WriteHear}{GitHub \faGithub}}}{}
% \resumeItemListStart
% \resumeItem{An invention I created during a co-op at Innovative Scientific Solutions Inc. that transcribes handwriting by listening to the sounds produced by the writing, implemented in MATLAB. \bref{https://github.com/jam643/WriteHear/blob/master/WriteHear.pdf}{The invention} won a U.S. Air Force wide competition.}
% \resumeItemListEnd


\resumeProjectHeading
{\textbf{Balancing Robot} $|$ \emph{\bref{https://github.com/jam643/BalancingBot}{GitHub \faGithub}}}{}
\resumeItemListStart
\resumeItem{A balancing cart-pole type robot, using cascaded PID control and complementary filter for state estimation. I designed and built the hardware and electronics and wrote the software.}
\resumeItemListEnd


\small{\item{\textbf{Additional Projects:}
              \bref{https://github.com/jam643/SailboatSim3D}{Autonomous Sailboat 3D Dynamics Simulator} (MATLAB),
              \bref{https://github.com/jam643/ColorFollowingBot}{Color Following Robot with Articulated Camera} (Python, OpenCV),
              \bref{https://github.com/jam643/DoublePendulumBlender}{Aesthetically Rendered Double Pendulum Simulation} (Blender/Python API),
              \bref{https://github.com/jam643/WriteHear}{WriteHear: Transcribe handwriting from sound} (MATLAB),
              \bref{https://github.com/jam643/BouncingBallSim}{Bouncing Ball Simulation} (C++),
              \bref{https://github.com/jam643/GameOfLife}{2D Geometric Shooter Game} (Python)}}

\resumeSubHeadingListEnd


%-------------------------------------------
\end{document}