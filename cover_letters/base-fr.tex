% Constants for phrase choices and expressions (greetings, thanks, etc)


\opening{\iflanguage{french}{Madame, Monsieur,}{Dear Sir or Madam,}}
\closing{\iflanguage{french}{Dans l'attente de votre retour, je vous prie d'agréer, Madame, Monsieur, mes salutations distinguées.}{Yours faithfully,}}

\newcommand{\Company}{}
% recipient data
\recipient{\phantom{.}}{}
\date{}
% \enclosure[Attached]{curriculum vit\ae{}}          % use an optional argument to use a string other than "Enclosure", or redefine \enclname
\makelettertitle{}

Concevoir des systèmes d'IA alliant rigueur technique et impact sociétal résume mon parcours: du développement d'environnements d'apprentissage par renforcement à Seoul National University à la revitalisation de la vie culturelle étudiante. Fraîchement diplômé, je suis passionné par les défis où l'exigence technique rencontre la créativité, et je souhaite mettre cette double expertise à votre service.

Au sein du SNU Vision \& Learning Lab, j'ai conçu un environnement de RL réduisant la surcharge d'évaluation des tâches à 1,7 \% du temps d'exécution, tout en formant des chercheurs aux workflows JAX/Flax (25 \% d'adoption en un an). Au-delà de la technique, j'ai dirigé des projets transverses: création d'outils open source comme RL-THOR (Python/Rust) pour démocratiser la recherche en IA, enseignement à 150+ étudiants, ou gestion de compétitions sportives nationales pour 1 800 participants. Ces expériences m'ont appris à concilier innovation et pragmatisme, en intégrant les contraintes opérationnelles dès la phase de conception.

Ce qui m'anime? Résoudre des problèmes complexes où l'ingénierie scalable rencontre des enjeux réels. Mon parcours internationale a forgé une approche globale: pour moi, les solutions d'IA doivent être aussi inclusives que performantes. RL-THOR en témoigne: son architecture hybride (Python pour la flexibilité, Rust pour l'efficacité) permet aux chercheurs de se concentrer sur l'innovation sans sacrifier la robustesse.

Je serais ravi d'échanger sur la façon dont mes compétences en Python, Rust et conception de systèmes IA/ML pourraient soutenir vos ambitions.

\makeletterclosing
