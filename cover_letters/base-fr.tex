% Constants for phrase choices and expressions (greetings, thanks, etc)


\opening{\iflanguage{french}{Madame, Monsieur,}{Dear Sir or Madam,}}
\closing{\iflanguage{french}{Dans l'attente de votre retour, je vous prie d'agréer, Madame, Monsieur, mes salutations distinguées.}{Yours faithfully,}}

\newcommand{\Company}{}
% recipient data
\recipient{\phantom{.}}{}
\date{}
% \enclosure[Attached]{curriculum vit\ae{}}          % use an optional argument to use a string other than "Enclosure", or redefine \enclname
\makelettertitle{}
Construire des systèmes à la fois rigoureux sur le plan technique et responsables sur le plan humain a toujours guidé mon parcours — de la recherche en apprentissage par renforcement à l'Université Nationale de Séoul à la revitalisation de la vie artistique sur le campus en tant que président d'association.

Ingénieur récemment diplômé, je m'épanouis à l'intersection entre exigence technique et créativité. A SNU Vision \& Learning Lab, j'ai conçu un environnement de d'apprentissage par renforcement réduisant les coûts d'évaluation à 1,7\% du temps d'exécution, tout en initiant mes collègues chercheurs à l'écosystème de machine learning JAX (adopté par un quart du labo en un an). En parallèle, j'ai conçu des outils open source comme RL-THOR, encadré plus de 150 étudiants, et coordonné des compétitions étudiantes nationales pour 1 800 participants.

Ce qui m'anime ? Résoudre des problèmes où l'ingénierie scalable rencontre la complexité du réel. Mon parcours entre la France et la Corée du Sud m'a appris à conjuguer rigueur scientifique, diversité culturelle et sens du collectif — autant d'atouts pour contribuer à des projets d'IA ambitieux et responsables.

Je serais ravi d'échanger sur la manière dont mes compétences en Python, Rust et conception de systèmes IA/ML peuvent servir votre mission.

\makeletterclosing
