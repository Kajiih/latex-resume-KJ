% Constants for phrase choices and expressions (greetings, thanks, etc)


\opening{\iflanguage{french}{Madame, Monsieur,}{Dear Sir or Madam,}}
\closing{\iflanguage{french}{Dans l'attente de votre retour, je vous prie d'agréer, Madame, Monsieur, mes salutations distinguées.}{Yours faithfully,}}

\newcommand{\Company}{Qualibat}
% recipient data
\recipient{\RecruitmentTeam{\Company}}{\Company\\Paris\\ France}
\date{}
% \enclosure[Attached]{curriculum vit\ae{}}          % use an optional argument to use a string other than "Enclosure", or redefine \enclname
\makelettertitle{}
Concevoir des systèmes d'IA robustes, scalables et utiles, c'est ce qui anime mon parcours d'ingénieur, entre recherche académique, développement open source et encadrement d'équipes. C'est avec enthousiasme que je postule au poste d'ingénieur Machine Learning chez Qualibat.

Diplômé de l'École des Mines de Saint-Étienne et de Seoul National University, j'ai travaillé au SNU Vision & Learning Lab sur la conception d'environnements d'apprentissage par renforcement optimisés, réduisant les coûts d'évaluation à 1,7\% du runtime. J'ai aussi encadré plus de 150 étudiants et accompagné l'adoption de JAX dans le laboratoire. Mes projets personnels (comme RL-THOR) témoignent de mon intérêt pour les architectures efficaces et la mise à disposition d'outils réutilisables par la communauté.

Je suis particulièrement motivé par les applications de l'IA conversationnelle, domaine en forte évolution, et je suis désireux de contribuer à la mise en place de solutions modernes et fiables, basées sur des LLM et des frameworks comme LangChain.

Autonome, curieux et structuré, je serais heureux de mettre mes compétences en Python, systèmes IA/ML et conception logicielle au service de vos projets.

\makeletterclosing
