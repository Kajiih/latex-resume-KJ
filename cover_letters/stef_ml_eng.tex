% Constants for phrase choices and expressions (greetings, thanks, etc)


\opening{\iflanguage{french}{Madame, Monsieur,}{Dear Sir or Madam,}}
\closing{\iflanguage{french}{Dans l'attente de votre retour, je vous prie d'agréer, Madame, Monsieur, mes salutations distinguées.}{Yours faithfully,}}

\newcommand{\Company}{STEF}
% recipient data
\recipient{\RecruitmentTeam{\Company}}{\Company\\Theix\\France}
% \date{22 avril 2025}
% \enclosure[Attached]{curriculum vit\ae{}}          % use an optional argument to use a string other than "Enclosure", or redefine \enclname
\makelettertitle
Construire des systèmes IA fiables, déployables et responsables est au cœur de mon parcours d'ingénieur. Je serais heureux de contribuer à la mise en production de modèles et d'applications de machine learning chez STEF, acteur clé de la logistique alimentaire en Europe.

Diplômé de l'École des Mines de Saint-Étienne et de Seoul National University, j'ai travaillé a SNU Vision \& Learning Lab sur des environnements d'apprentissage par renforcement performants, tout en promouvant de nouvelles pratiques d'ingénierie IA auprès de mes collègues chercheurs. J'ai également conçu des outils open source comme RL-THOR, encadré plus de 150 étudiants et participé à la création de pipelines d'expérimentation IA reproductibles et scalables.

Maîtrisant Python, Docker, Git, FastAPI, et sensibilisé aux enjeux MLOps, je suis à l'aise avec les problématiques de mise en production de modèles dans des environnements cloud ou hybrides. Je suis curieux des usages autour des LLMs et du prompt engineering, et motivé à approfondir ces domaines dans un cadre industriel exigeant comme le vôtre.

Travailler au sein de votre pôle Data, à l'intersection de l'innovation technologique et de la logistique responsable, me paraît être un excellent terrain pour allier impact concret et excellence technique.

\makeletterclosing
