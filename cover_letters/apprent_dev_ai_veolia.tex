Madame, Monsieur,

Concevoir des systèmes IA robustes, responsables et utiles à la société est au cœur de mon engagement d'ingénieur. C'est pourquoi je souhaite rejoindre SARPI en tant qu'alternant développeur Python IA.

Diplômé de l'École des Mines de Saint-Étienne et de Seoul National University, je poursuis aujourd'hui ma formation par une alternance de 15 mois en ingénierie logicielle. Passionné par le NLP, le deep learning et la recherche appliquée, j'ai notamment développé un environnement d'apprentissage par renforcement optimisé au Vision & LearningLab de SNU, ainsi que des outils open source comme RL-THOR*. J'ai également encadré plus de 150 étudiants sur des projets de machine learning et NLP, et suis à l'aise avec l'ensemble des technologies mentionnées dans votre offre (Python, Pytorch, modèles préentraînés, bases de données, visualisation...).

Travailler sur des projets à impact environnemental dans un cadre industriel me motive particulièrement. Je serais heureux de contribuer activement au déploiement de vos solutions IA, tout en développant mes compétences au contact de vos équipes.

Je serais ravi d'échanger avec vous à ce sujet.

Veuillez agréer, Madame, Monsieur, mes salutations distinguées.

Julian Paquerot

* https://github.com/Kajiih/rl_thor
